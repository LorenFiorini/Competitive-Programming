\documentclass[11pt]{article}
\usepackage{colortbl}
\usepackage{setspace}
\usepackage{amsfonts}
\usepackage{breqn}
\usepackage{tabto}
\author{Fiorini Heredia Lorenzo Andrés}
\title{Sets}
\date{2021.09.17}

\begin{document}
\maketitle{Section 1.1 exercises - Answers}
\setstretch{1.1}

\begin{enumerate}

% 1 
\item Which of the following are sets? \\ 
{
	(a) is not a set. \\
	(b) is not a set. \\
	(c) is not a set. \\
	(d) is a set. \\
	(e) is a set.
}

% 2
\item Let $S = \{-2, -1, 0, 1, 2, 3\}$. Describe each of the following sets as $\{x \in S : p(x)\}$, where $p(x)$ is some
condition on $x$.\\ 
{
	(a) $A = \{ x \in S : x > 0 \}$\\
	(b) $B = \{ x \in S : x \geq 0\}$\\
	(c) $C = \{ x \in S : x < 0\}$\\
	(d) $D = \{ x \in S : |x| = 2 \hspace{2pt}\lor\hspace{2pt}  x = max(S)\}$\\
}

% 3
\item  Determine the cardinality of each of the following sets: \\
{
	(a) $|A| = 5$\\
	(b) $|B| = 11$\\
	(c) $|C| = 51$\\
	(d) $|D| = 2$\\
	(e) $|E| = 1$\\
	(f) $|F| = 2$\\
}

% 4
\item Write each of the following sets by listing its elements within braces.\\
{
	(a) $A = \{-3, -2, -1, 0, 1, 2, 3, 4 \} $\\
	(b) $B = \{-2, -1, 0, 1, 2 \} $\\
	(c) $C = \{ 1, 2, 3, 4 \} $\\
	(d) $D = \{ 0, 1 \} $\\
	(e) $E = \{  \} $\\
}


% 5
\item Write each of the following sets in the form $\{x \in \mathbb{Z} : p(x)\}$, where $p(x)$ is a property concerning $x$.\\ 
{
	(a) $A = \{ x \in \mathbb{Z} : x < 0 \}$\\
	(b) $B = \{ x \in \mathbb{Z} :  |x| \leq 3\}$\\
	(c) $C = \{ x \in \mathbb{Z} :  |x| \leq 2 \hspace{2pt}\land\hspace{2pt} x \neq 0\}$\\
}



% 6
\item The set $E = \{2x : x \in \mathbb{Z} \}$ can be described by listing its elements, namely $E = \{..., -4, -2, 0, 2, 4,...\}$.
List the elements of the following sets in a similar manner.\\
{
	(a) $A = \{..., -3, -1, 1, 3, 5, ... \} $\\
	(b) $B = \{..., -8, -4, 0, 4, 8, ...  \} $\\
	(c) $C = \{..., -5, -2, 1, 4, 7, ...  \} $\\
}


% 7
\item The set $E = \{..., -4, -2, 0, 2, 4,...\}$ of even integers can be described by means of a defining condition
by $E = \{y = 2x : x \in \mathbb{Z} \}=\{2x : x \in \mathbb{Z} \}$. Describe the following sets in a similar manner.\\
{
	(a) $A = \{y = 3x + 2 : x \in \mathbb{Z} \}=\{3x + 2 : x \in \mathbb{Z} \}$ \\
	(b) $B = \{y = 5x : x \in \mathbb{Z} \}=\{5x : x \in \mathbb{Z} \}$ \\
	(c) $C = \{y = x^{3}  : x \in \mathbb{N} \}=\{x^{3}  : x \in \mathbb{N} \}$ \\
}


% 8
\item (a) Describe the set A by listing its elements.\\
{
	\tab \hspace{1cm} $A = \{ -3, -2, 2, 3\}$\\
	(b) Give an example of three elements that belong to B but do not belong to A.\\
	\tab \hspace{1cm} $ \frac{5}{2} \hspace{2pt},\hspace{4pt}  \frac{7}{2} \hspace{2pt},\hspace{4pt}   4  $\\
	(c) Describe the set C by listing its elements.\\
	\tab \hspace{1cm} $ C = \{ \sqrt{2}, 2\}$\\
	(d) Describe the set D in another manner.\\
	\tab \hspace{1cm} $ D = \{ x \in \mathbb{Q} : x = 2\}$\\
	(e) Determine the cardinality of each of the sets A,C and D\\
	\tab \hspace{1cm} $|A| =  4$\\
	\tab \hspace{1cm} $|C| =  2$\\
	\tab \hspace{1cm} $|D| =  1$\\
}

% 9
\item Determine C.\\
{
	$C = \{10, 13 \} $\\
}




\end{enumerate}



\end{document}