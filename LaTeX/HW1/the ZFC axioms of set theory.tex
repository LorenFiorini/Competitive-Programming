\documentclass[12pt]{article}
\usepackage{colortbl}
\usepackage{setspace}
\usepackage{breqn}
\author{Fiorini Heredia Lorenzo Andrés}
\title{The Zermelo-Fraenkel axioms of set theory}
\date{2021.09.17}

\begin{document}
\maketitle
\setstretch{1.1}

\begin{enumerate}

% 1 Extensionality
\item {\color{blue}Axiom of Extensionality} If $X$ and $Y$ have the same elements, then $X = Y$.
\begin{equation}
	\forall u   \hspace{2pt}  
	(u \in X \equiv i \in Y)  
	\Rightarrow X = Y
\end{equation}

% 2 Unordered Pair
\item {\color{blue}Axiom of the Unordered Pair} For any $a$ and $b$ there exists a set $(a, b)$ that contains exactly $a$ and $b$.
\begin{equation}
	\forall \hspace{1pt} a \hspace{2pt}  
	\forall \hspace{1pt} b  \hspace{2pt}  
	\exists \hspace{1pt} c  \hspace{2pt}  
	\forall \hspace{1pt} x   \hspace{2pt}  
	(x \in c \equiv (x = a \lor x = b))
\end{equation}

% 3 Subsets
\item {\color{blue}Axiom of Subsets}  if $\varphi$ is a property with parameter $p$, the for any $X$ and $p$ there exists a set $Y = \{u \in X: \varphi(u, p)\}$ that contains all those $u \in X$ that have the property $\varphi$.
\begin{equation}
	\forall \hspace{1pt} X  \hspace{2pt}  
	\forall \hspace{1pt} p  \hspace{2pt}  
	\exists \hspace{1pt} Y  \hspace{2pt}  
	\forall \hspace{1pt} u   \hspace{2pt}  
	(u \in Y \equiv (u \in X \land x = b))	
\end{equation}

% 4 Sum Set
\item {\color{blue}Axiom of Sum Set} For any $X$ there exists a set $Y = {\cup} X$, the union of all elements of $X$.
\begin{equation}
	\forall \hspace{1pt} X  \hspace{2pt}  
	\exists \hspace{1pt} Y  \hspace{2pt}  
	\forall \hspace{1pt} u   \hspace{2pt}  
	(u \in Y \equiv \exists \hspace{1pt} z \hspace{2pt}  (z \in X \land \varphi (u, p)))	
\end{equation}


% 5 Power Set
\item {\color{blue}Axiom of Power Set} For any $X$ there exists a set $Y = P (X)$, the set of all subsets of $X$.
\begin{equation}
	\forall \hspace{1pt} X  \hspace{2pt}  
	\exists \hspace{1pt} Y  \hspace{2pt}  
	\forall \hspace{1pt} u   \hspace{2pt}  
	(u \in Y \equiv u \subseteq X)	
\end{equation}


% 6 Infinity
\item {\color{blue}Axiom of Infinity} There exists an infinite set.
\begin{equation}
	\exists \hspace{1pt} S \hspace{2pt}  
	[\O \in S \land (\forall \hspace{1pt} x\in S) 
	[x \cup \{x\} \in S]]
\end{equation}


% 7 Replacement
\item {\color{blue}Axiom of Replacement} If $F$ is a function, then for any $X$ there exists a set $Y = F[X] = \{F(x): x \in X\}$.
\begin{equation}
\begin{split}
	\forall \hspace{1pt} x  \hspace{2pt}  
	\forall \hspace{1pt} y  \hspace{2pt}  
	\forall \hspace{1pt} z  \hspace{2pt}  
	[\varphi (x, y, p) \land \varphi (x, z, p) \Rightarrow y = z] 
	\\ \Rightarrow
	\forall \hspace{1pt} X  \hspace{2pt}  
	\exists \hspace{1pt} Y  \hspace{2pt}  
	\forall \hspace{1pt} y  \hspace{2pt}  
	[y \in Y \equiv \exists \hspace{1pt} x \in X \varphi (x, y, p)]
\end{split}
\end{equation}


% 8 Foundations
\item {\color{blue}Axiom of Foundations} Every nonempty set has an $\in$-minimal element.
\begin{equation}
	\exists \hspace{1pt} S \hspace{2pt}  
	[ S \neq \O \Rightarrow 
	(\exists \hspace{1pt} x  \in S)
	\hspace{2pt}  S  \cap x = \O ]
\end{equation}


% 9 Choice
\item {\color{blue}Axiom of Choice}  Every family of nonempty sets has a choice function.
\begin{equation}
	\forall x  \in a \hspace{3pt}  
	\exists \hspace{1pt} A(x,y) \Rightarrow 
	\exists \hspace{1pt} y  \hspace{2pt}  
	\forall \hspace{1pt} x \hspace{2pt} 
	\in a A(x, y(x))
\end{equation}

\end{enumerate}



\end{document}