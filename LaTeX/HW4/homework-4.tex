\documentclass[12pt]{article}

\usepackage{setspace}
%\usepackage{blindtext}
%\usepackage{microtype}
\usepackage{enumitem}
\usepackage{amsmath}
\usepackage{amssymb}

\begin{document}
\title{\textbf{Homework 4}}
\author{Fiorini Heredia Lorenzo Andrés}
\date{2021.10.13}
\maketitle
\setstretch{1.1}

\begin{enumerate}

% 1
\item {\large State the definition of what it means for the style of encodings to be isomorphic, and prove that they are in fact isomorphic.}
	
	 Isomorphism is a one-to-one mapping between two sets with the same number of elements or cardinality, which is invertible and follows particular operations. Left association and right association of an ordered triple is isomorphic. Consider left association as $((a,b),c)$ and right association as $(a,(b,c))$. Notice that both structure are exactly the same: both has the same numbers of elements in pairs, 3 particular elements which are in ordered expression, and if both are inversed, they will be exactly the same.  What's left is just their associativity: which 2 ordered elements should be considered first. However, regardless the difference, the final order of triple will not change whatsoever. \\


% 2
\item {\large Write formulas of $R^T$ and $R^C$!}
\begin{itemize}
	\item $R^T$ is the reversal of $R$. If set $X$ is in binary relation $R$ with set $Y$, then $\forall x\in X, y\in Y: R^T=\{(y, x) \,|\, xRy\}$ is the reversal of relation $R$.
	\item $R^C$ is the complement of $R$.  If set $X$ is in binary relation $R$ with set $Y$, then $\forall x\in X, y\in Y: R^C=\not{R}=\{(x,y) \,|\, \neg xRy\}$ is the complement of relation $R$.\\
\end{itemize}


% 3
\item {\large 9.1 Exercises}
\begin{enumerate}
\setcounter{enumi}{4}
	% a
	\item 
	 Let $A = \{a, b, c\}$ and $B = \{1, 2, 3, 4\}$. Then $R_1 = \{(a, 2), (a, 3), (b, 1),
	  (b, 3), (c, 4)\}$ is a relation from $A$ to $B$, while $R_2 = \{(1, b), (1, c), (2, a), (2, b), (3, c), (4, a), (4, c)\}$ is a relation from $B$ to $A$. A relation $R$ is defined on A by $xRy$ if there exists $z \in B$ such that $x R_1 z$ and $z R_2 y$. Then, relation $R=\{(a,a),(a,b),(a,c),(b,b),(b,c),(c,a),(c,c)\}$.\\
	% b
	\item
	For the relation $R = \{(x,y) : x \leq y\}$ defined on $\mathbb{N}$, $R^{-1}=\{(y,x) : y\geq x \text{ for } y,x\in\mathbb{N}\}$.\\
	% c
	\item
	Let $A=\{1,2,3,4\}$. To satisfy $R\cap R^{-1} = \O$,  $A\times A=\{(1,1),(2,2),
	(3,3),(4,4)\}$ must be omitted because these will result in nonempty sets. Thus, $|A\times A|=4^2-4=12$, and the total relations on $A$ will be $2^{12}=4,096$. \\
	
\end{enumerate}


% 4
\item {\large 9.2 Exercises}
\begin{enumerate}

\setcounter{enumi}{5}
	% a
	\item
	Let $S = \{a, b, c\}$. Then $R = \{(a, a), (a, b), (a, c)\}$. Since $(b,b),(c,c)\notin R$, it is not reflexive. Since $(b,a),(c,a)\notin R$, it is also not symmetric. For transitivity, consider 2 cases. First, two ordered pairs $(a,a),(a,b)\in R$ and $(a,b)\in R$. Second, $(a,a),(a,c)\in R$ and $(a,c)\in R$. We shall ignore other cases where there is no required pairs to be considered transitive. Thus, $R$ is transitive.\\
	
	% b
	\item
	Let $A=\{a,b,c,d\}$. Relation $R=\{(a,b),(a,d),(b,c),(b,d)\}$ has none of the following properties: reflexive, symmetric, transitive. It is not reflexive because $(a,a),(b,b),(c,c),(d,d)\notin R$. It is not symmetric because $(b,a),(d,a),(c,b),(d,b)\notin R$. It is not transitive because $(a,b),(b,c)\in R$ but $(a,c)\notin R$.\\

	% c
	\item
	Let $A=\{a,b,c,d\}$. Relation $R$ is reflexive if it has ordered pairs of $(a,a),(b,b),(c,c),(d,d)$. $R$ is symmetric if we consider the inverse of given required pairs, namely $(b,a),(c,b),(d,c)$. If we combine those two properties, we get the minimum number of elements $R$ should have in order to satisfy the conditions, namely $(a,a),(a, b),(b,a),(b,b),(b,c),\\(c,b),(c,c),(c,d),(d,c)(d,d)$. Lastly, $R$ is transitive if we add $(a,c),(a,d),(b,d),(c,a),(d,a),(d,b)$. Thus, the only possibility of finding such solution is just 1 relation which contains every single ordered pairs that can be made from set $A$.\\

	% d
	\item
	\begin{enumerate}
		\item $R_1=\{(1,1),(1,2),(2,1),(2,2),(2,3),(3,2),(3,3),(4,4)\}$ is reflexive and symmetric but not transitive.
		\item $R_2=\{(1,1),(1,2),(1,3),(2,2),(2,3),(3,3),(4,4)\}$ is reflexive and transitive but not symmetric.
		\item $R_3=\{(1,2),(2,1),(1,1),(2,2)\}$ is symmetric and transitive but not reflexive.
		\item $R_4=\{(1,1),(1,2),(2,2),(2,3),(3,3),(4,4)\}$ is reflexive but neither symmetric nor transitive.
		\item $R_5=\{(1,2),(2,1),(3,4),(4,3)\}$ is symmetric but neither reflexive nor transitive.
		\item $R_6=\{(1,2),(2,3),(1,3)\}$ is transitive but neither reflexive nor symmetric.\\
	\end{enumerate}

	% e
	\item
	Let $A=\{a,b,c\}$. If relation R is not reflexive, then it only has a maximum of 2 ordered pairs, namely $(a,a),(b,b)$. If it's not symmetric, then it only has a maximum of 5 ordered pairs, namely $(a,b),(a,c),(b,a),(b,c),(c,a)$. If we combine those ordered pairs, we got $(a,a),(a,b),(a,c),(b,a),(b,b),(b,c),(c,a)$. To make it not transitive, we need to omit one of the pairs which satisfies the property of transitive. Thus, the maximum number of elements in a relation R on a 3-element set with none of 3 properties is 6.\\

	% f
	\item
	A relation $R$ is defined on $S$ by $pRq$ if $p$ and $q$ have a root in common. Take example $p = (x - 1)^2$ and $q = x^2 - 1$ have the root 1 in common so that $pRq$. Notice that the relation holds for $pRp$, because p and p itself have the same root (also works for q). Consequently, R is reflexive. Also, for $pRq$, it also holds for $qRp$, because they share the same root as long as both p and q are in relation. Thus, R is also symmetric. Lastly, if $pRq$ is true, then $p$ and $q$ share the same root. However, if $qRr$ is true, $q$ and $r$ share the same root, but it might occur that $p\not{R}r$ because their root might be not the same, as $S$ is the set of polynomials of degree atmost 3, in which an element can contain moremore than 1 root. Thus, $R$ is not transitive.
\end{enumerate}


\end{enumerate}

\end{document}